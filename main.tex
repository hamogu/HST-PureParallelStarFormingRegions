%%%%%%%%%%%%%%%%%%%%%%%%%%%%%%%%%%%%%%%%%%%%%%%%%%%%%%%%%%%%%%%%%%%%%%%%%%
%
%    phase1-GO.tex  (use only for General Observer and Snapshot proposals; use phase1-AR.tex for Archival Research and
%                      Theory proposals use phase1-DD.tex for GO/DD proposals and use phase1-MC.tex for GO/MC rapid response
%                       proposals).
%
%    HUBBLE SPACE TELESCOPE
%    PHASE I OBSERVING PROPOSAL TEMPLATE 
%     FOR CYCLE 25 and beyond
%
%    Version 2.1; November 2018
%
%    Guidelines and assistance
%    =========================
%     Cycle 27 Announcement Web Page:
%
%         https://hst-docs.stsci.edu/ 
%
%    Please contact the STScI Help Desk if you need assistance with any
%    aspect of proposing for and using HST. Either send e-mail to
%    help@stsci.edu, or call 1-800-544-8125; from outside the United
%    States, call [1] 410-338-1082.
%
%
%%%%%%%%%%%%%%%%%%%%%%%%%%%%%%%%%%%%%%%%%%%%%%%%%%%%%%%%%%%%%%%%%%%%%%%%%%%

% The template begins here. Please do not modify the font size from 12 point.

\documentclass[12pt]{article}
\usepackage{phase1}
\usepackage{graphicx}
\usepackage{natbib}
\usepackage{graphicx}
%\usepackage{sidecap}
\include{journals}
%\bibpunct{(}{)}{;}{a}{}{,}

\begin{document}
\section{Title, Co-Is, etc.}
\textbf{This information will go into the form, the pdf itself must be anonymous. I'm just collecting it here so we have it all in one place for now.}

\textbf{Title:} A pure-parallel search for faint stuff in star forming regions (acronym: PPS-FS-SF)

\pagebreak
%   1. SCIENTIFIC JUSTIFICATION
%       (See https://hst-docs.stsci.edu/display/HSP/HST+Cycle+27+Preparation+of+the+PDF+Attachment)
%
%
\justification          % Do not delete this command.
% Enter your scientific justification here. 
Stars and planetary systems continue to form in our Milky Way today. Images of these star forming regions are among the most spectacular images ever taken by the Hubble Space Telescope (HST). Star forming regions contain structure over a large range of scales. The clouds themselves are typically several parsecs across; in each cloud dozens to thousands of stars form. They collapse into proto-stars which are still deeply embedded into a dusty envelope and can be seen only in the sub-mm and IR; infall onto those protostars happens though a circumstellar disk and these early objects often also drive powerful jets that propagate through the envelope and pierce the cloud. Once the envelope depletes, the stars become visible in the optical, but accretion from the disk continues. At the same time, grains in the disks coagulate into planets. Our understanding of disks, jets, and the remaining envelope around young stars depends on high-resolution images such as Fig.~\ref{fig:krist}.


\begin{figure}
    \centering
    \includegraphics[height=8cm]{Krist98.png}
    \caption{FS Tauri in low-mas star forming regions with WFPC2. The image is combined from several exposures mostly in the F675W filter. Average exposure time is about 800~s. The image reveals a hour-glass shaped reflection nebula and a jet. From: \citet{1998ApJ...501..841K}}
    \label{fig:krist}
\end{figure}


Most stars have low-masses and late spectral types and in low-mass star forming regions the most massive object might be a G-type star. In these regions, jets are supposed to be weak and are not supposed to to have the power to propagate through the cloud material for long distances. High-resolution images are usually taken around objects known to be ``interesting'' and that biases our understanding, for example, we know that jet emission is time variable, yet imaging to detect jets is done mostly on objects with known outflows today. A blind survey of random patches in a star forming cloud could tell us which fraction of the area or volume has outflow material propagating though it - an important check for models that try to predict if the injected turbulence is sufficient to disperse the cloud.

In high-mass star forming regions, irradiation by the more massive and much brighter members of the cluster has the potential evaporate the cloud significantly; the bright stars also eject more powerful winds and outflows and they can photo-evaporate the disks or jets of the surrounding stars (Fig.~\ref{fig:reiter}). Again, high-resolution imaging tends to concentrate on the brightest and most active stars and thus does not give an honest picture of the typical conditions in the star forming regions.

\begin{figure}
    \centering
    \includegraphics[width=.45\textwidth]{reiter13_fig3.png}
    \includegraphics[width=.54\textwidth]{carina_xray_label.jpg}
    \caption{\emph{Left:}Protostellar jets in the Carina region, observed with WFC3 in same filters we propose to use; exposure time about 2000~s for H$\alpha$ and about 3000~s for [O~{\sc iii}]. From: \citet{2013MNRAS.433.2226R}} \emph{Right:} X-ray view of the entire Carina star forming region. This image is about 1 degree across and shows that practically any location for a WFC3 or ACS pointing in the star forming region will show young stars with disks (Chandra discovered $> 14,000$ source) and ionized plasma. Image credit: NASA/CXC/PSU/L.Townsley et al.
    \label{fig:reiter}
\end{figure}


In X-rays, many of the massive star forming regions in the Milky Way (e.g.\ \citealt{2011ApJS..194....1T} for Carina, \citealt{2010ApJ...713..871W} for Cyg OB2, \citealt{2005ApJS..160..379F} for the Orion Nebula Cloud, \citealt{2008AJ....135..693W} for RCW 108) and a surprising number of the lower-mass star forming regions (e.g.\ \citealt{2018AJ....155..241W} for Serpens, \citealt{2012AJ....144..101G} for IRAS 20050+2720, \citealt{2007A&A...468..353G} for the Taurus Molecular Cloud, using \emph{XMM-Newton} because of its larger field-of-view) have been surveyed with \emph{Chandra}, which gives us spatial resolution on sub-arcsecond scales. In the IR, essentially all star forming regions have been looked at with \emph{Spitzer} (e.g.\ \citealt{2009ApJS..184...18G} surveyed all star forming regions within 1 kpc, examples for more massive, further regions are XXX). The IR and the hard X-rays are less extincted than optical wavelength, and thus allow us to view deeper into the embedded regions. The combination of X-rays and IR information is a great tool to identify the point soures in clusters and to establish cluster membership. From the spectral energy distribution of a source we can infer the presence of an accretion disk, if an IR excess over the SED of the star itself is present \citep{2009ApJS..184...18G}. Essentially all young stars are active \citep{2018ApJS..235...43T}, and thus an X-ray detection is a good way to distinguish cluster members from reddened background galaxies or older foreground stars \citep{2013ApJS..209...32B}. However, stellar clusters consist of more than point sources and in fact much of the physics that determines the evolution of a cluster depends on the interplay between the gas and dust in the cloud itself. The cool dust below a few hundred K is visible in the radio as dust continuum emission or in molecular emission lines; slightly warmer dust can be seen in the IR. However, there are many processes that heat material above the dust sublimation temperature and at that point it becomes invisible in the IR. Those processes are among the most energetic and interesting ones in star forming regions: Irradiation by hot and bright massive stars, disk evaporation, stellar winds, jet emission and shock fronts in the cloud material. Only the most energetic processes (supernova explosions, O star winds) reach temperatures that are not accessible in optical observations any longer, these can be seen as a faint X-ray continuum emission (Fig.~\ref{fig:reiter}), but low signal means that cannot be studied at high spatial resolution.

We propose a pure-parallel survey in the optical with \emph{HST}. In a pure parallel observation, we do not specify individual targets, but simply switch on the WFC3 or ACS cameras when \emph{HST} is observing any target in a star forming region with STIS or COS; we will see essentially random fields a few arminutes away from the primary spectroscopic target. Thanks to ULYSSES, an initiative using 500-1000 orbits of director's time for UV spectroscopy of young stars over the next three years, we know that a large number of orbits will be allocated to spectroscopic targets in star forming regions in addition to what the TAC recommends.

Our program can address the following science questions:

\textbf{Where does this text go?} Radio observations are well-suited to survey large regions on the sky and to trace wide-angle molecular outflows \citep[e.g.][]{2003MNRAS.341..707C} but they miss the faster, more collimated outflow components, which originate deeper in the gravitational potential, i.e.\ closer to the proto-star \citep{2003ApJ...590L.107A}.

Our proposed observing strategy is very similar to the atlas of protoplanetary disks in the Orion Nebula by \citet{2008AJ....136.2136R}. They survey covers about 450~arcmin$^2$ in $B$, $V$, H$\alpha$, $I$, and $z$ with average exposure times about 400~s. They detect about 3200 compact sources and 219 sources with circumstellar matter. The largest fraction of those (178) are externally ionized disk in emission; 36 disks are seen in extinction against the bright nebula in the background. Five objects show jets without an apparent disk (meaning that the disk is likely to be smaller because the objects are more evolved). Figure~\ref{fig:ONCACS} shows examples of these detections. 

\begin{figure}
    \centering
    \includegraphics[width=\textwidth]{ONCACS.png}
    \caption{Examples of different objects detected in an ACS survey of the Orion Nebula Cloud by \citet{2008AJ....136.2136R}, each column shows one object in two different filters (\emph{top}: F775W is SDSS $i'$, \emph{bottom}: F658N covers the H$\alpha$ line). For each category, two objects are shown. Note how some features stand out in the emission line, while others are visible in the broad-band.}
    \label{fig:ONCACS}
\end{figure}

\textbf{Jet sources and jet launching}
It is unknown how jets are launched and collimated, but ultimately, they have to be powered by the gravitational energy released in the accretion process. In close-by stars, jets are layered like an onion where slower, outer layers are surrounded by denser and faster layers inside \citep{2000ApJ...537L..49B}. The innermost layers can reach a few 1000~km/s for jets from massive stars and heat up to X-ray emitting temperatures, but the bulk of the mass is neutral or moderately ionized and emits H$\alpha$, or optical emission lines such as [O~{\sc i}] 6300\AA{} or [O~{\sc iii}] 5007\AA{} (Fig.~\ref{fig:reiter}, left). The sample of well-studied jets is surprisingly small, especially for later stages of star formation and lower mass sources where jets are less powerful and thus fainter harder to find; such micro-jets are sometimes only a few arcseconds long, with the brightest part of the emission close to the star. From the ground, those jets can only be seen with adaptive optics imaging and will not be found in large-scale surveys; but they are easily visible in HST images. Even for more powerful jets, which are resolved at resolutions typical for ground-based seeing, we often require high-resolution imaging to identify the jet source.

\textbf{Knots in jets}
Jets are not homogenous structures, but contain knots. These can be due to mass-ejection events in the past, and, since mass ejection and accretion are related, contain a fossil archive of previous accretion rates \citep{2014A&A...563A..87E}. However, as knots move away from the source, they become fainter. If jet emission switches on an off in a source, there may be no indication of a jet seen in the star today; thus surveys looking for jets would not target those sources, however, fossil records of previous jet event may still be out there in the form of faint emission regions, at some distance form the source. HST imaging in a blind search in H$\alpha$ and [O~{\sc i}] is a promising way to find those features, if the exisit. This is actually easier in low-mass star forming regions where the source density if lower and thus the likely jet source can be identified, even if emission is not seen close to it.

\textbf{Disk evolution and evaporation}

Understanding external photoevaporation is also important for the history of the solar systems and the formation of our Earth \citep{2015ApJ...815..112K}.

Theoretically, we expect that the livetimes of disks depend strongly on the formation environment, where massive stars in dense clusters should evaporated disks within a Myrs or so and close encounters can strip additional material from the disks \citep[e.g.][]{2004ApJ...611..360A,2019MNRAS.485.1489W,2019arXiv190211094N}but there are recent indications that external photo-evaporation can be important even in low-mass star forming regions \citep{2017MNRAS.468L.108H}.
There are two ways to study this observationally: Look how disk properties in a massive cluster change with distance from the most massive members \citep{2018ApJ...860...77E}, or compare clusters of different mass. Our proposed study can help with both, giving us high-resolution observations of low-mass star forming regions where normally one would not invest much time on a blind survey to find disks in imaging, and images of massive clusters, a little further away from the objects usually studied.




\textbf{Multiplicity}
A large fraction of all young stars forms in binaries or higher order multiples, yet many questions about the evolution of multiple systems remain unsolved. First, simply knowing the fraction of multiple systems and the mass and radius distribution of companions can tell us under what conditions companions grow in the envelops of disk of the primary (like plants) versus how the envelope forms two stars. If we can image disks or outflows of at least on of the stars, we can also start to probe if orbital plane and individual disks are aligned.

For close-in binaries, the companion will obviously change the evolution of the primary's disk, e.g. by triggering episodic accretion \citep{2013ApJ...766...62G }, and thus possibly reducing the time and the mass reservoir available for planet formation. For binaries with separations comparable to the size of a disk in the T Tauri phase (a few hundred au) this is less clear. In the well-studied case of RW~Aur~A and B a tidal stream connects the disks \citep{2006A&A...452..897C}. However, to study these effects in a sample, we first need an accurate census of multiplicity and how this might change with the mass of the central star and its formation environment.

For all these reasons, we need to identify more systems with companions, which are typically fainter than the primary stars, and one of the best methods to do that is imaging from space in red bands or the near-IR where the contrast between the primary and a lower mass secondary is less extreme than in, e.g.\ the $V$ band. 

\citet{2001A&A...379..147D} find about 20\% of their sample in the young cluster NGC 6611 have a companion in the range 200-3000~au, where the companion has at least a tenth of the mass of the primary. With HST's sub-arcsecond PSF, we can detect bright companions at 200~au for star forming regions at 1 kpc, and we can perform photometry on companions separated by 500~au or so at that distance. 


For many sources, few examples, survey find more, but don't know what


Star forming regions are studied in a wide range of wavelengths from the X-rays to the radio with surveys spanning hundreds of square degrees down to high-resolution spectroscopy of individual objects. Typically, the large scale surveys are used to identify individual jets or stars to be studied in more detail. However, there is a crucial gap in this procedure: Large-area surveys usually do not have the spatial resolution to detect faint emission close to brighter sources (e.g. lower-mass companions or small jets) and do not go deep enough to detect the faint features, in particular if they are specially extended. \textbf{We propose a blind survey of star forming regions as a pure-parallel observation program to close this gap.}



In a request for 100 pure-parallel obits we will sample about 0.3 sq deg of star forming cloud. This is several orders of magnitude less than the large \emph{Spitzer} surveys of star forming regions, but it will still be the largest dataset available with the superb spatial resolution that HST offers and the number of filters we plan to use. Ground-based based surveys are typically only done in broad-band filters (IPHAS REF HERE, which includes also an H$\alpha$ filter is the exception) and without adaptive optics, thus the space-based \emph{HST} images that we will obtain will be superior in spatial resolution, depth, and coverage of spectral features.

Since this request is for a pure-parallel program, where we piggy-back onto observations done with SITS or COS for other programs, this survey will cost zero primary HST orbits. The downside is that we cannot know a-priory how many disks, outflows or cloud feature we will image and how many low-mass stars we will find. If the primary targets are in dense star forming regions such as the Orion Nebula Cloud or Carina, we are virtually guaranteed to see several young stars, cloud structures, disks and outflows for any target location; on the other hand, if the primary targets are in close-by low-density star forming regions such as Taurus, we may find some field to devoid of cluster members. When STScI identifies any primary targets in Phase II that this program could piggy-back on, we will check existing surveys to select the regions with the largest potential to see ``interesting'' objects for the science case laid out. This means that fields closer to the center of more massive star forming regions receive priority over fields covering the outer regions of lower-mass star forming regions.

Pure-parallel surveys have taken deep images in star forming regions before and have been used for extragalactic astronomy REF HERE but also to search for ultracool objects such as L and T dwarfs in the galaxy \citep{2005ApJ...631L.159R}. These studies show the potential for serendipitous science from deep high-resolution imaging beyond the science justification we lay out here. We analyzed some of this data and verified that structures of the molecular cloud are indeed visible in great detail; however, all those data are taken in wide filters only while we propose additional observations in narrow band filter which probe particular conditions in irradiated clouds, disks and jets.

%%%%%%%%%%%%%%%%%%%%%%%%%%%%%%%%%%%%%%%%%%%%%%%%%%%%%%%%%%%%%%%%%%%%%%%%%%%

%   2. DESCRIPTION OF THE OBSERVATIONS
%       (See https://hst-docs.stsci.edu/display/HSP/HST+Cycle+27+Preparation+of+the+PDF+Attachment)
%
%
\describeobservations   % Do not delete this command.
% Enter your observing description here.
We are requesting observations in pure parallel mode, meaning that our images will be taken when some other program uses COS or STIS are prime instruments and happens to look at a star forming region. Thanks to the ULLYSES program, we know that a few dozen orbits of spectroscopic data will be taken in star forming regions, in addition to any GO programs for spectroscopy of young stars that may be approved in this cycle by the TAC. In Phase II, STScI will generate a list of such primary orbits, which may be suitable for the execution of the observations we propose for here, but as a pure-parallel program, we have no influence over the setup of those observations. That means that the exact pointing, orientation and number of dither positions will be determined by the primary science program. Thus, we describe a general set-up of the observations here, but details will have to be adjusted in Phase II.

To reach our science objectives, we request to obtain images in two broadband (SDSS $r'$ and $i'$) and two narrow-band filters (H$\alpha$ and an oxygen line). In high-mass star forming regions, [O~{\sc iii}] isa good tracer of hot nebular gas, but in low-mass star forming regions, [O~{\sc iii}] 5007\AA{} is weak or absent. The WFC3 has an [O~{\sc i}] 6300\AA{} filter, while the ACS/WFC does not and we propose to use the [O~{\sc i}] 6300\AA{} filter with the WFC3 for targets in low-mass star forming regions. Table~\ref{tab:setup} summarizes the filters to be used.
For a typical usable orbit time of about 2800~s after acquisition, we can take one exposure of 540~s in each filter. Once we know the exact duration of the orbit (which depends on the position and roll angle specified by the primary target), we will optimize the exposure times to reduce the time lost to buffer dumps and filter changes.

If we have only one orbit with a specific pointing, we will perform imaging with the WFC3 because it is located closer to the STIS or COS FOV. In a star forming region, a STIS or COS observation is most likely centered on a bright, active star and thus observing closer to it increases the chances to find irradiated cloud or outflow features. 

\begin{table}[htbp]
    \centering
    \begin{tabular}{c|ccc}
    \hline\hline
           & WFC3 & WFC3 & ACS/WFC \\
        star forming region & high-mass & low-mass & any\\
        \hline
        SDSS $r'$ & F625W & F625W & F625W \\
        SDSS $i'$ & F775W & F775W & F775W\\
        H$\alpha$ & F656N & F656N & F658N\\{}
        [O~{\sc iii}] 5007\AA{} & F502N & -- & F502N\\{}
        [O~{\sc i}] 6300\AA{}& -- & F631N & -- \\
        \hline
    \end{tabular}
    \caption{Filters to be used for observations in high and low-mass star forming regions.}
    \label{tab:setup}
\end{table}


Commonly, spectroscopic observations last for more than one orbit and thus it is likely that we will have access to more than one parallel orbit with almost the same pointing. In this case, we will use the second orbit for observations with ACS/WFC (see table~\ref{tab:setup}). ACS has a FOV a few arcmin away from WFC3. Should more than two orbits be available with the same pointing, we will repeat the WFC3 and ACS observations to detect fainter features and to take advantage of multiple dither positions (if the primary observation dithers) and to obtain more exposures in each filter to help with cosmic ray rejection. 

We looked at STIS and COS observations in star forming regions as well as previous pure-parallel programs in the archive and find that it is not uncommon to have 5-10 orbits with very similar pointings; which would give us a summed exposure time of about one orbit per filter. For longer observations, HST will often observe the primary target at different roll angles and thus WFC3 and ACS will see a different area on the sky even if the primary program observes the same target for a large number of orbits. Adding up the FOV for WFC and ACS, we think that with a request of 100 pure parallel orbits we will sample 300-400 sq arcmin deg of star forming region (assuming on average 3-5 orbits with the same pointing) - not a large area, but since this is a pure parallel program it does not cost any primary orbits to perform such a survey.

Just to give some examples for our sensitivity: In a single orbit (exposure 540~s in each filter), we can detect an M7 star in a close star forming region (130~pc) or an M2 at 0.5~kpc; sources like this should also be part of 2MASS, giving us fluxes in five colors so that we can fit the spectral type, extinction, and check for the presence of a disk in the spectral energy distribution (SED) \citep{2007ApJS..169..328R}.
See the figures in the text above for examples of resolved cloud or jet emission features that were found in observations comparable to what we propose here.

Note that this is a request for a blind survey. Since the targets of the primary observations are not known, we do not know what number of features we will find. Obviously, more observing times will allow us to cover a larger area or go deeper (depending on the distribution of targets for the primary observations). At the same time, this program is still useful if only a smaller number than the requested 100 orbits is granted or available; in this case we will simply survey a smaller area (or not as deep, depending on the distribution of primary targets).



%%%%%%%%%%%%%%%%%%%%%%%%%%%%%%%%%%%%%%%%%%%%%%%%%%%%%%%%%%%%%%%%%%%%%%%%%%%

%   3. SPECIAL REQUIREMENTS
%       (See https://hst-docs.stsci.edu/display/HSP/HST+Cycle+27+Preparation+of+the+PDF+Attachment)
%
%
\specialreq             % Do not delete this command.
% Justify your special requirements here, if any.
None.

%%%%%%%%%%%%%%%%%%%%%%%%%%%%%%%%%%%%%%%%%%%%%%%%%%%%%%%%%%%%%%%%%%%%%%%%%%%

%   4. COORDINATED OBSERVATIONS
%       ((See https://hst-docs.stsci.edu/display/HSP/HST+Cycle+27+Preparation+of+the+PDF+Attachment)
%
%
\coordinatedobs          % Do not delete this command.
% Enter your coordinated observing plans here, if any.
None.

%%%%%%%%%%%%%%%%%%%%%%%%%%%%%%%%%%%%%%%%%%%%%%%%%%%%%%%%%%%%%%%%%%%%%%%%%%%

%   5. JUSTIFY DUPLICATIONS
%       (See https://hst-docs.stsci.edu/display/HSP/HST+Cycle+27+Preparation+of+the+PDF+Attachment)
%
%
\duplications           % Do not delete this command.
% Enter your duplication justifications here, if any.
If a region on the sky that would we available to this program has been observe before with these or similar filters, the new observations still add additional data to look for fainter objects, or, for brighter objects, to study their proper motion (e.g\ the expansion of a gas bubble or the motion of a know in a jet).

%%%%%%%%%%%%%%%%%%%%%%%%%%%%%%%%%%%%%%%%%%%%%%%%%%%%%%%%%%%%%%%%%%%%%%%%%%%
\bibliographystyle{aa} % style aa.bst
\bibliography{biblio}


\end{document}          % End of proposal. Do not delete this line.
                        % Everything after this command is ignored.
